\documentclass[tikz,border=10pt]{standalone}
\usepackage[utf8]{inputenc}
\usepackage[T1]{fontenc}
\usepackage{tikz}
\usetikzlibrary{shapes.geometric, arrows.meta, positioning, fit, backgrounds}

\begin{document}

\tikzstyle{proceso} = [rectangle, rounded corners, minimum width=5cm, minimum height=1cm, text centered, text width=5cm, draw=black, fill=blue!30]
\tikzstyle{proceso2} = [rectangle, rounded corners, minimum width=5cm, minimum height=1cm, text centered, text width=5cm, draw=black, fill=blue!50]
\tikzstyle{proceso3} = [rectangle, rounded corners, minimum width=5cm, minimum height=1cm, text centered, text width=5cm, draw=black, fill=blue!70, text=white]
\tikzstyle{iteracion} = [rectangle, rounded corners, minimum width=3cm, minimum height=0.8cm, text centered, text width=3cm, draw=black, fill=white]
\tikzstyle{fase} = [rectangle, minimum width=7cm, minimum height=0.8cm, text centered, text width=7cm, draw=black, fill=black, text=white, font=\bfseries]
\tikzstyle{mantenimiento} = [rectangle, rounded corners, minimum width=5cm, minimum height=1cm, text centered, text width=5cm, draw=black, fill=blue!40]
\tikzstyle{nuevas} = [rectangle, rounded corners, minimum width=5cm, minimum height=1cm, text centered, text width=5cm, draw=black, fill=green!30]
\tikzstyle{arrow} = [thick,->,>=Stealth]
\tikzstyle{arrow-cycle} = [thick,->,>=Stealth,red!70,line width=1.5pt]

\begin{tikzpicture}[node distance=0.6cm]

% Nodos principales
\node (contexto) [proceso] {Contexto y análisis del problema de investigación};
\node (reconocimiento) [proceso, below=of contexto] {Reconocimiento de requerimientos};
\node (especificacion) [proceso2, below=of reconocimiento] {Especificación de requerimientos};
\node (diseno) [proceso2, below=of especificacion] {Diseño de software};
\node (implementacion) [proceso3, below=of diseno] {Implementación y codificación};

% Fase de iteraciones
\node (prueba) [iteracion, below=1.2cm of implementacion, xshift=-1.5cm] {Prueba de\\software};
\node (compilacion) [iteracion, right=0.5cm of prueba] {Compilación e\\instalación};
\node (fase-label) [fase, below=0.1cm of implementacion] {Fase de iteraciones};
\node (correccion) [iteracion, below=0.6cm of fase-label] {Corrección de errores y\\recodificación};

% Crear fondo amarillo para la fase de iteraciones
\begin{scope}[on background layer]
\node[fill=yellow!40, rounded corners, fit=(prueba)(compilacion)(correccion)(fase-label), inner sep=0.4cm] {};
\end{scope}

% Nodos finales
\node (instalacion) [proceso2, below=1.5cm of correccion] {Instalación-verificación-\\desinstalación};
\node (git) [mantenimiento, below=of instalacion] {Mantenimiento vía git};
\node (nuevas) [nuevas, below=of git] {Desarrollo de nuevas versiones};

% Flechas principales
\draw [arrow] (contexto) -- (reconocimiento);
\draw [arrow] (reconocimiento) -- (especificacion);
\draw [arrow] (especificacion) -- (diseno);
\draw [arrow] (diseno) -- (implementacion);
\draw [arrow] (implementacion) -- (fase-label);
\draw [arrow] (fase-label) -- (prueba);
\draw [arrow] (fase-label) -- (compilacion);
\draw [arrow] (prueba) -- (correccion);
\draw [arrow] (compilacion) -- (correccion);
\draw [arrow] (correccion) -- (instalacion);
\draw [arrow] (instalacion) -- (git);
\draw [arrow] (git) -- (nuevas);

% Flechas de ciclo de iteración
\draw [arrow-cycle] (correccion.west) .. controls +(-2,1) and +(-2,-1) .. (prueba.west);
\draw [arrow-cycle] (correccion.east) .. controls +(2,1) and +(2,-1) .. (compilacion.east);

% Flecha de ciclo grande (nuevas versiones a fase de iteraciones)
\draw [arrow] (nuevas.west) .. controls +(-3,0) and +(-3,0) .. (correccion.west);

\end{tikzpicture}

\end{document}
